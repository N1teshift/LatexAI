\begin{multicols}{2}
\setlength{\columnseprule}{0.4pt}
\def\columnseprulecolor{\color{black}}
\begin{uzd}
Copy this table. not
Tick (\checkmark) the correct boxes.

\phantom{a}

$\begin{array}{|c|c|c|}
\hline
\text{Number} & \text{Rational} & \text{Irrational} \\
\hline
\sqrt{36} & & \\
\hline
\sqrt{48} & & \\
\hline
\sqrt{64} & & \\
\hline
\sqrt{84} & & \\
\hline
\sqrt[3]{100} & & \\
\hline
\end{array}$
\end{uzd}

\begin{uzd}
Look at these numbers:

$12.77, \ -36, \ \sqrt{27}, \ \sqrt{500}, \ \tfrac{61}{12}, \ -\sqrt[3]{8}$

\begin{itemize}
\item[(a)] Write the \textbf{irrational numbers}.
\item[(b)] Write the \textbf{integers}.
\end{itemize}
\end{uzd}

\begin{uzd}
Write whether each of these numbers is an integer or a surd.

\resetcolumnrule
\begin{multicols}{3}
\text{(a)} $\sqrt{25}$

\text{(b)} $\sqrt[3]{25}$

\text{(c)} $\sqrt{125}$

\text{(d)} $\sqrt[3]{125}$

\text{(e)} $\sqrt{225}$

\text{(f)} $\sqrt[3]{225}$
\end{multicols}
\end{uzd}


\begin{uzd}
Is each of these numbers rational or irrational? Give a reason for each answer.

\resetcolumnrule
\begin{multicols}{2}
\text{(a)} $\sqrt{3} + 6$

\text{(b)} $\sqrt{3+6}$

\text{(c)} $\sqrt{64} + \sqrt[3]{64}$

\text{(d)} $\sqrt[3]{8} + \sqrt[3]{19}$
\end{multicols}
\end{uzd}

\begin{uzd}
\phantom{a}

\begin{itemize}
\item[(a)] Find $1.5^2$.
\item[(b)] Show that $\sqrt{2.25}$ is a rational number.
\item[(c)] Is $\sqrt{20.25}$ a rational number? Give a reason for your answer.
\item[(d)] Is $\sqrt{1.331}$ a rational number? Give a reason for your answer.
\end{itemize}
\end{uzd}

\begin{uzd}
Without using a calculator, show that

\begin{itemize}
\item[(a)] $3 < \sqrt[3]{41} < 4$  
\item[(b)] $9 < \sqrt[3]{800} < 10$  
\item[(c)] $1.1 < \sqrt{1.36} < 1.2$
\end{itemize}
\end{uzd}



\begin{uzd}
Without using a calculator, find an irrational number between

\resetcolumnrule
\begin{multicols}{3}
\text{(a)} $2$ and $3$

\text{(b)} $6$ and $7$

\text{(c)} $1.4$ and $1.5$
\end{multicols}
\end{uzd}


\begin{uzd}
Without using a calculator, estimate

\begin{itemize}
\item[(a)] $\sqrt{140}$ to the nearest integer  

\item[(b)] $\sqrt[3]{350}$ to the nearest integer
\end{itemize}
\end{uzd}

\columnbreak

\begin{uzd}
Arun says:


\smartbox{
My calculator shows $2 \tfrac{7}{81} = 2.086419753$ and this does not have a repeating pattern, so $2 \tfrac{7}{81}$ is irrational.}

\begin{itemize}
\item[(a)] Is Arun correct? Give a reason for your answer.
\item[(b)] Do you think $\sqrt{2 \tfrac{7}{81}}$ is a rational number? Give a reason for your answer.
\end{itemize}
\end{uzd}

\begin{uzd}
\phantom{a}

\begin{itemize}
\item[(a)] Use a calculator to show that $\sqrt{2} \times \sqrt{32}$ is a rational number.
\item[(b)] Find two irrational numbers with a product of
\resetcolumnrule
\begin{multicols}{3}
(i) $6$ \\ 
(ii) $9$ \\ 
(iii) $10$
\end{multicols}
\end{itemize}
\end{uzd}

\begin{uzd}
\phantom{a}

\begin{itemize}
\item[(a)] Explain why $5 + \sqrt{2}$ is an irrational number.
\item[(b)] Find two irrational numbers with a sum of $5$.
\item[(c)] Explain why it is impossible to find two rational numbers with a sum of $\sqrt{5}$.
\item[(d)] Is it possible to find two rational numbers with a product of $\sqrt{5}$? Give a reason for your answer.
\end{itemize}
\end{uzd}

\begin{uzd}
This Venn diagram shows all the numbers from a number line.  
$A$ is the set of integers.
$B$ is the set of rational numbers.  
Copy the diagram and put each of these numbers in the correct place.  

$25, \ 5.5, \ \tfrac{5}{19}, \ \sqrt{25}, \ \sqrt[3]{25}$
\end{uzd}

\begin{uzd}
If $n = 20$, find the value of

\begin{itemize}
\item[(a)] \begin{itemize}
  \item[(i)] $\sqrt{n} + 2$  
  
  \item[(ii)] $\sqrt{n} - 2$  
  
  \item[(iii)] $(\sqrt{n} + 2)(\sqrt{n} - 2)$
  \end{itemize}
\item[(b)] Sofia says:  

\smartbox{
If $n$ is an integer, then $(\sqrt{n} + 2)(\sqrt{n} - 2)$ is also an integer.}

Is Sofia correct?
Give some evidence to support your answer.
\end{itemize}
\end{uzd}
\end{multicols}