% Document type configurations
% This file defines different document types and their settings

% ===== PRESENTATION SETTINGS =====
\newcommand{\setupPresentation}{
    \documentclass[aspectratio=169,10pt]{beamer}
    \usepackage[utf8]{inputenc}
    \usepackage[T1]{fontenc}
    \usepackage[lithuanian]{babel}
    \usepackage{amsmath,amssymb,mathtools}
    \usepackage{graphicx}
    \usepackage{tikz}
    \usepackage{pgfplots}
    \usepackage{tabularx,array,multicol,multirow,xcolor}
    \usepackage{enumitem}
    \usepackage{hyperref}
    \pgfplotsset{compat=1.18}
    \usetikzlibrary{calc,arrows.meta,angles,quotes,intersections,decorations.pathmorphing}
    \graphicspath{{content/images/}}
    
    % Lists & minor defaults
    \setlist[itemize]{leftmargin=1.2em,itemsep=0.25em}
    \setlist[enumerate]{leftmargin=1.4em,itemsep=0.25em}
    \setbeamertemplate{navigation symbols}{}
}

% ===== MATH THEORY & WORKSHEETS SETTINGS =====
\newcommand{\setupMathTheory}{
    \documentclass[8pt,a4paper]{report}
    \usepackage[utf8]{inputenc}
    \usepackage[T1]{fontenc}
    \usepackage[lithuanian]{babel}
    \usepackage{amssymb, amsmath, amsthm}
    \usepackage{lmodern}
    \usepackage{textgreek}
    \usepackage[top=0.25cm, bottom=0.25cm, left=0.25cm, right=0.25cm, footskip=1cm, a4paper]{geometry}
    \usepackage{fncychap}
    \usepackage{fancyhdr}
    \usepackage{graphicx}
    \usepackage{tabularx}
    \usepackage{array}
    \usepackage{tikz}
    \usepackage{multicol}
    \usepackage{multirow}
    \usepackage{xcolor}
    \usepackage{pgfplots}
    \usepackage{hyperref}
    \usepackage{enumitem}
    
    \usetikzlibrary{angles,quotes,calc}
    \usetikzlibrary{calc, decorations.pathmorphing}
    \usetikzlibrary{intersections}
    \usetikzlibrary{arrows.meta}
    \usetikzlibrary{arrows}
    
    \def\RPlus{\rule[0.165em]{.5em}{.165em}\hspace{-.33em}\rule[0em]{.165em}{.5em}\,}
    \graphicspath{ {content/images/} }
    
    \makeatletter
    \renewcommand{\DOCH}{%                                     
        \CNoV\thechapter \space \CNV\FmN{\@chapapp} \par\nobreak
        \vskip 40\p@}
    \makeatother
    
    \renewcommand*{\proofname}{$\bigtriangleup$}
    \linespread{1.5}
    \numberwithin{table}{chapter}
    \numberwithin{figure}{chapter}
    \pagestyle{plain}
    
    \newtheorem{teorema}{teorema}
    \newtheorem{teorema1}{teorema}
    \theoremstyle{definition}
    \newtheorem{uzd}[teorema]{uždavinys}
    \newtheorem{puzd}[teorema]{papildomas uždavinys}
    \newtheorem{ap}[teorema]{apibrėžimas}
    \newtheorem{teig}[teorema]{teiginys}
    \newtheorem{pas}[teorema]{pastaba}
    \newtheorem{pav}[teorema]{pavyzdys}
    \newtheorem{sav}[teorema]{savybė}
    
    \def\R{\mathbb{R}}
    \def\N{\mathbb{N}}
    \def\Q{\mathbb{Q}}
    \def\Z{\mathbb{Z}}
    \def\I{\mathbb{I}}
    
    \newcommand{\lk}[1]{\glqq #1\grqq\,}
    
    \pgfmathdeclarefunction{cbrt}{1}{%
      \pgfmathparse{#1>=0 ? (#1)^(1/3) : -((abs(#1))^(1/3))}%
    }
}

% ===== ARTICLE SETTINGS =====
\newcommand{\setupArticle}{
    \documentclass[11pt,a4paper]{article}
    \usepackage[utf8]{inputenc}
    \usepackage[T1]{fontenc}
    \usepackage[lithuanian]{babel}
    \usepackage{amsmath,amssymb,mathtools}
    \usepackage{graphicx}
    \usepackage{tikz}
    \usepackage{pgfplots}
    \usepackage{tabularx,array,multicol,multirow,xcolor}
    \usepackage{enumitem}
    \usepackage{hyperref}
    \usepackage[margin=2.5cm]{geometry}
    
    \pgfplotsset{compat=1.18}
    \usetikzlibrary{calc,arrows.meta,angles,quotes,intersections,decorations.pathmorphing}
    \graphicspath{{content/images/}}
    
    % Common math definitions
    \def\R{\mathbb{R}}
    \def\N{\mathbb{N}}
    \def\Q{\mathbb{Q}}
    \def\Z{\mathbb{Z}}
    \def\I{\mathbb{I}}
    
    \newcommand{\lk}[1]{\glqq #1\grqq\,}
}
